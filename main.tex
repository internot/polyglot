\documentclass[10pt, conference, compsocconf]{IEEEtran}

\title{Crossing Origins by Crossing Formats}

\begin{document}

\author{\IEEEauthorblockN{Jonas Magazinius}
\IEEEauthorblockA{Department of Computer Science\\
Chalmers University of Technology\\
Gothenburg, Sweden\\
jonas.magazinius@chalmers.se}
\and
\IEEEauthorblockN{Billy Rios}
\IEEEauthorblockA{dept. name of organization\\
name of organization, acronyms acceptable\\
City, United States of America\\
Email: name@xyz.com}
\and
\IEEEauthorblockN{Andrei Sabelfeld}
\IEEEauthorblockA{Department of Computer Science\\
Chalmers University of Technology\\
Gothenburg, Sweden\\
andrei@chalmers.se}
}

\maketitle



\begin{abstract}

\end{abstract}



\begin{IEEEkeywords}
Web Security; Polyglot; Cross-domain;

\end{IEEEkeywords}








%1 pages
\section{Introduction}
\label{sec:intro}

General introduction for web application security.

The following attacks have  been particularly successful (support with references).

\subsection{Injection attacks}

Injection attacks have gone from classical to sophisticated.

New breed of attacks.

\subsection{Content repurposing attacks}

Server side verfication

Examples


Bridge to the next session saying that we take these attacks to the
next level.

%3 pages
\section{Crossing origins by crossing formats}

Describe how the attacks lead to Same-Origin Policy bypass.

\subsection{Polyglots}

What and how?

Why PDF format?

\subsection{Exploiting polyglots}

A polyglot can be used for exploitation if there is a difference between how 
content is interpreted at the time of verification compared to time of use. 
It is possible to create such a polyglot by mixing content, that is benign when verified, with content, that is 
malicious when executed. The verification process will see the benign content 
and allow it, but subsequently the malicious content is executed.

In literature we find Content repurposing, content smuggling, chameleon, gifar

\subsection{Syntax injection}

Description of attack technique

Attack building blocks

Scenarios

\subsection{Polyglot upload}

Description of attack technique

Building blocks

Scenarios

%3 pages
\section{Vulnerabilities}


\subsection{Concept}



\subsubsection{The object-tag}

The object-tag allows the developer to specify which format 
the embedded should be interpreted as. 

\subsubsection{Syntax injection}

the minimal malformed structure that is required for Adobe 
Reader to consider it a PDF. 

[anything] \%PDF- [anything] \
[ anything ] [space] trailer [space] << [any name value pair] 
/Root [space ] <<[any name value pair] /Pages [space]<< [any name value pair] >> [any name value pair] >> [any name value pair] >>

This means the required tokens that we need to be able to 
inject are: \%PDF-, trailer, <<, >> and /.
The \%PDF and trailer token is usually no problem. From what 
I have seen, sometimes PDF is turned to lower case which 
prevents the attacks. Quite naturally, unless the site is 
vulnerable to traditional XSS, the << and >> tokens are usually 
converted to \&lt; or some other encoding. The exception is in 
a JavaScript string context. In this context only </script> 
pose a threat, in case the < is not encoded this is usually 
prevented by escaping the slash:

var s = 'user string<\\/script>';

This is bad from a PDF syntax point of view, because \\/Root is 
not valid. But </script> is only valid when the < is directly 
followed by /. So we can create a PoC which is not vulnerable to 
traditional XSS, where the injection is  in a string and only </ 
is escaped, which will be vulnerable to PDF injection.


\subsubsection{Polyglot upload}



\subsubsection{Cross-origin communication}

For communication with the containing page, one can either use 
hostContainer.postMessage(info) or app.launchURL('\#'+info). The 
latter has the advantage that it works in Firefox which seems 
to not handle the postMessage very well.


\subsection{Instances}

How this problem presents itself in various instances of browsers and readers.


\subsubsection{Browsers}

Vulnerable: Firefox, Safari, Opera, Chrome

Not vulnerable: Internet Explorer


\subsubsection{Readers}

Vulnerable: Adobe Reader

Not vulnerable: Chrome PDF-viewer

Not tested: Apple Preview









%3 pages
\section{Evaluation}


\subsection{Bypassing content filters}


\subsubsection{Server-side upload filters}

Filter can verify that the benign file format is indeed benign 
without ever noticing the malicious file format hidden within.


\subsubsection{Cross-site scripting filters}

This effectively bypasses any cross-site scripting filters such as 
NoScript or filters built into browsers. Since the Reader plugin 
handles the response, the broser never gets to see the content


\subsubsection{Context sensitive filters}

Clever filters that adapt their filtering to the context in which 
the user content is included. Basically filters that allow input 
that do not form harmful HTML. 


\subsection{Alexa top 100}

Alexa top 100 results

%Baidu.com (allows <>, but escapes /)
%LinkedIn (allows <>, but escapes /)
%Soso (allows <>, but escapes /)
%Youku (vulnerable to traditional XSS)
%Soku (vulnerable to traditional XSS)
%alibaba.com (allows <>, but escapes /)
%about.com (vulnerable to traditional XSS, http://linux.about.com/sitesearch.htm?q=XSS&SUName=--%3E%3Cimg+src=x+onerror=alert%281%29%3E)
%sogou.com (potential target http://www.sogou.com/web?query=%25PDF+%3C%3C+%2F+%3E%3E&_asf=www.sogou.com&_ast=1349689275&w=01019900&p=40040100&sut=11010&sst0=1349689274528)










%1-2 pages
\section{Mitigation}


\subsection{Server-side mitigation}


\subsubsection{Syntax injection}

In the case of PDF; always encode tokens related to the PDF 
syntax. Generally, hard to ensure that tokens for all potential 
file formats are properly encoded without breaking anything.


\subsubsection{Polyglot upload}

Sever content from a sandbox domain.


\subsection{Browser}

Prevent object-tag type attribute to override content type 
provided by server. 


\subsection{Reader}

Stricter parsing of PDF files.

Not render content served with incorrect content type.








%1 page
\section{Related work}







%1 page
\section{Conclusions}


We have contacted the vendors...

Mentioned that Adobe was notified long time ago.





%1-2 pages
\bibliographystyle{IEEEtran}
\bibliography{literature}


\end{document}